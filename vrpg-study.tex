\documentclass{article}
\setcounter{page}{1}
\title{VRPG for fine tunings goals and path finding}
\author{Saransh Sharma}

\begin{document}
	\maketitle
	
	\section{Introduction}
	The objective of this discourse is to explore the implications and feasibility of setting a financial target, such as accruing a monthly or annual income of \$100,000. In layman's terms, this goal may seem straightforward, yet it inevitably prompts further inquiry into its attainability. While one may construct a theoretical framework to achieve this, it does not necessarily ensure the successful realization of the goal in tangible terms. 

To illustrate this, consider the metaphor of a bird soaring freely in the sky. If this bird were to suddenly decide to cease its flight and instead aim to generate \$100,000 a month, it would necessitate a significant alteration in its lifestyle and behavior. This analogy serves to highlight the conditions and changes that are inherent in the pursuit of such financial objectives.The central questions this paper seeks to address are akin to this metaphor. What are the factors that drive us to set such goals, and how do we go about achieving them? What are the underlying motivations, and can they be succinctly summarized? The aim is to identify and analyze the variables that might define the objectives we strive to accomplish.This exploration is not limited to financial goals but extends to those that require skill or mental acuity. It is an attempt to articulate and understand the complexities of goal-setting and achievement in various aspects of human endeavor.

This discourse seeks to examine individuals who generate income significantly exceeding the previously mentioned figure of \$100,000 per annum or per month, and the strategies they employ to achieve this. Drawing upon the bird metaphor elucidated earlier, the bird must first acquire the skill to persuade others to part with such a substantial sum of money. Furthermore, the bird must possess sufficient motivation to alter its lifestyle. This prompts the question: what kind of bird would relinquish its freedom of flight for monetary gain?. Similarly, we can pose the same question in the context of human behavior. While the specifics of each individual's situation may vary, it is possible to discern a common structure in our lives, built around identifiable values or beliefs that guide our actions. 

In a hypothetical scenario where machines are tasked with generating income, one must consider the system that would enable such a function. What mechanisms would it employ, and how would it operate? While we have some understanding of the methods a human might use to generate income, the strategies a machine might employ remain an open question. This paper aims to explore these intriguing possibilities.
In the pursuit of income generation, a human would typically engage in a process of selecting a professional field that aligns with their skills and interests. This decision is often influenced by the current socio-economic climate, including factors such as job availability and market demand. 

The individual's level of experience plays a crucial role in this process, as does their ability to cultivate relationships, both professionally and personally. Furthermore, the capacity for self-motivation, as well as the ability to inspire others, is a significant factor in achieving success in one's chosen field.

Operating within any professional environment necessitates an understanding and application of various theoretical concepts. These concepts, which may encompass strategies, methodologies, or principles, underpin the individual's actions and decisions, thereby influencing their overall performance and success.

Unless one resorts to illicit activities such as theft, which in itself requires a certain skill set, it can be posited that there exist functions operating at a level where we act out of inherent or acquired abilities. Drawing parallels with the bird metaphor discussed earlier, just as the bird utilizes its wings and its ability to fly, humans leverage their cognitive abilities to think, act, and create tangible outcomes.

Commencing with a seemingly naive idea of generating a monthly income of \$100,000 might appear far-fetched. However, when viewed through the lens of human potential and the capacity for innovation, it becomes a plausible objective. This premise underscores the remarkable ability of humans to transform abstract ideas into concrete realities, thereby highlighting the limitless possibilities inherent in the human endeavor.

\section{A lifetime advice}

During my residency in Pune, I had the opportunity to engage in a profound conversation with a friend who provided financial advice. Despite not experiencing any significant life issues at the time, his counsel on problem-solving strategies regarding life's challenges proved to be insightful. He proposed a unique approach, which, though not fully implemented, I believe holds potential to be formalized into a systematic method.

This approach was not immediately apparent to me; it took approximately four years for its potential to fully dawn on me. The method he suggested was as follows: firstly, articulate your goal, which could range from the realistic, such as acquiring a new skill, to the fantastical, such as learning to fly. Secondly, identify the obstacles associated with achieving this goal. For instance, in the case of the aspiration to fly, the lack of wings would be a primary impediment. While the number of potential obstacles could be infinite, it is crucial to focus on those most pertinent to the goal at hand.

The next step involves defining the values associated with the goal. This can be a complex task, given the nebulous nature of values. It requires a consideration of both current values and those that would be necessary to achieve the desired goal. He suggested that these values could be categorized into two types: qualitative and quantifiable.

Subsequently, he advised me to list the resources that could aid in achieving the goal. Resources, in this context, are elements that can facilitate the goal's realization, such as knowledge or skills. For instance, understanding the principles of aerodynamics could be a valuable resource for someone aspiring to fly.

Finally, he recommended that I document all these elements and iterate over them repeatedly. This process, he suggested, could lead to a deeper understanding of one's true desires.

Intrigued by this unconventional approach from a financial advisor, I inquired about its origin. He informed me that it was his original concept, and to his knowledge, no similar methodology existed elsewhere.

I spent several years looking for the same idea or a concept somewhere on the internet the one that I found recently is something that has similar name but differs in techique\footnote{https://support.infrontanalytics.com/en/support/solutions/articles/77000430396-what-is-gprv-} that I am about to work out. I realised later that what if these variables could be utilised for a profound path finding or could be used in application on data or some kind of algortihem that we can utilise for resolving say paths but it was harder to put out in something that could be implemted in generalised problem solving. I realised that this method could be useful in many cases for solving and why i keep mentioning that path finding meaning if someone is stuck at some problem or that has some form of a goal associated a problem wihout a goal is not a problem then.

I tried writing this down in different forms and some I am going to present you to make sure we are sticking to the practical side of it without getting too much in technicality. I am not sure if i need to qoute some literature or something before, I am not a conventional researcher. 

\section{VRPG as solver}

Consider a practical scenario wherein you aspire to purchase a car in 2023. This endeavor necessitates financial expenditure for the car's purchase, fuel, and maintenance. Furthermore, you may be environmentally conscious and concerned about the carbon dioxide emissions from the car. This example, while commonplace, is relevant to many individuals. Applying the VRPG (Values, Resources, Problems, Goals) model to this situation may present some complexities, but for the sake of clarity and understanding, let's proceed with defining some concepts and conventions.

Let's denote:

	\[ 
	G= Goals 
	\]
	\[ 
	P = Set \ of Problems 
	\]
	\[ 
	R = Set \ of Resources 
	\]
	\[ 
	V = Set \ of Values 
	\]

With these variables defined, we can assign values to them. For instance, 

	\[ 
	G= The \ aspiration \ to  \ purchase  \ a  \ car \ in \ 2023 
	\]

The next step involves identifying potential problems. These could include factors such as limited parking space or narrow roads, particularly if you reside in a place like Portugal. Another potential problem could be the lack of driving skills, which is a plausible issue in Portugal, where only 459 out of 1000 people own cars.

Thus, we have:

	\[ 
	P= {P(1), P(2)..P(n)} 
	\]
	
	\[ 
	P= Lack \ of \ driving \ skills, limited \ parking \ space , financial \ constraints
	\]

Problems can be both qualitative and quantitative. Qualitative problems lack measurable attributes, while quantitative problems can be measured. For instance, limited parking space and financial constraints could fall under both categories.

Next, we identify the resources available. These could include financial means such as savings, loans, or credit. It's important to note that these variables can be dynamic and may interchange with problems or values.

	\[ 
	R= {R(1), R(2)..R(n)} 
	\]
	
	\[ 
	R= Stable \ monthly\ income, \ parental \ support, \ ability\  to \ work
	\]

Finally, we consider the values that influence our decision. For instance, despite having the means to purchase a car, you might be committed to reducing carbon emissions and not contributing to the proliferation of vehicles on the planet.

	\[ 
	V= {V(1), V(2)..V(n)} 
	\]
	
	\[ 
	V= Desire \ to \ reduce \ CO2 \ emissions, \ reluctance \ to \ contribute \ to \ vehicle \ proliferation. 
	\]

This comprehensive framework allows us to analyze and understand the various factors that influence our decisions and actions.


\end{document}