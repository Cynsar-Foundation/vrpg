\documentclass{article}
\usepackage{amsmath}
\usepackage{listings}
\setcounter{page}{1}
\title{VRPG for fine tunings goals and path finding}
\author{Saransh Sharma}

\begin{document}
	\maketitle
	
	\section{Introduction}
	The objective of this discourse is to explore the implications and feasibility of setting a financial target, such as accruing a monthly or annual income of \$100,000. In layman's terms, this goal may seem straightforward, yet it inevitably prompts further inquiry into its attainability. While one may construct a theoretical framework to achieve this, it does not necessarily ensure the successful realization of the goal in tangible terms. 

To illustrate this, consider the metaphor of a bird soaring freely in the sky. If this bird were to suddenly decide to cease its flight and instead aim to generate \$100,000 a month, it would necessitate a significant alteration in its lifestyle and behavior. This analogy serves to highlight the conditions and changes that are inherent in the pursuit of such financial objectives.The central questions this paper seeks to address are akin to this metaphor. What are the factors that drive us to set such goals, and how do we go about achieving them? What are the underlying motivations, and can they be succinctly summarized? The aim is to identify and analyze the variables that might define the objectives we strive to accomplish.This exploration is not limited to financial goals but extends to those that require skill or mental acuity. It is an attempt to articulate and understand the complexities of goal-setting and achievement in various aspects of human endeavor.

This discourse seeks to examine individuals who generate income significantly exceeding the previously mentioned figure of \$100,000 per annum or per month, and the strategies they employ to achieve this. Drawing upon the bird metaphor elucidated earlier, the bird must first acquire the skill to persuade others to part with such a substantial sum of money. Furthermore, the bird must possess sufficient motivation to alter its lifestyle. This prompts the question: what kind of bird would relinquish its freedom of flight for monetary gain?. Similarly, we can pose the same question in the context of human behavior. While the specifics of each individual's situation may vary, it is possible to discern a common structure in our lives, built around identifiable values or beliefs that guide our actions. 

In a hypothetical scenario where machines are tasked with generating income, one must consider the system that would enable such a function. What mechanisms would it employ, and how would it operate? While we have some understanding of the methods a human might use to generate income, the strategies a machine might employ remain an open question. This paper aims to explore these intriguing possibilities.
In the pursuit of income generation, a human would typically engage in a process of selecting a professional field that aligns with their skills and interests. This decision is often influenced by the current socio-economic climate, including factors such as job availability and market demand. 

The individual's level of experience plays a crucial role in this process, as does their ability to cultivate relationships, both professionally and personally. Furthermore, the capacity for self-motivation, as well as the ability to inspire others, is a significant factor in achieving success in one's chosen field.

Operating within any professional environment necessitates an understanding and application of various theoretical concepts. These concepts, which may encompass strategies, methodologies, or principles, underpin the individual's actions and decisions, thereby influencing their overall performance and success.

Unless one resorts to illicit activities such as theft, which in itself requires a certain skill set, it can be posited that there exist functions operating at a level where we act out of inherent or acquired abilities. Drawing parallels with the bird metaphor discussed earlier, just as the bird utilizes its wings and its ability to fly, humans leverage their cognitive abilities to think, act, and create tangible outcomes.

Commencing with a seemingly naive idea of generating a monthly income of \$100,000 might appear far-fetched. However, when viewed through the lens of human potential and the capacity for innovation, it becomes a plausible objective. This premise underscores the remarkable ability of humans to transform abstract ideas into concrete realities, thereby highlighting the limitless possibilities inherent in the human endeavor.

\section{A lifetime advice}

During my residency in Pune, I had the opportunity to engage in a profound conversation with a friend who provided financial advice. Despite not experiencing any significant life issues at the time, his counsel on problem-solving strategies regarding life's challenges proved to be insightful. He proposed a unique approach, which, though not fully implemented, I believe holds potential to be formalized into a systematic method.

This approach was not immediately apparent to me; it took approximately four years for its potential to fully dawn on me. The method he suggested was as follows: firstly, articulate your goal, which could range from the realistic, such as acquiring a new skill, to the fantastical, such as learning to fly. Secondly, identify the obstacles associated with achieving this goal. For instance, in the case of the aspiration to fly, the lack of wings would be a primary impediment. While the number of potential obstacles could be infinite, it is crucial to focus on those most pertinent to the goal at hand.

The next step involves defining the values associated with the goal. This can be a complex task, given the nebulous nature of values. It requires a consideration of both current values and those that would be necessary to achieve the desired goal. He suggested that these values could be categorized into two types: qualitative and quantifiable.

Subsequently, he advised me to list the resources that could aid in achieving the goal. Resources, in this context, are elements that can facilitate the goal's realization, such as knowledge or skills. For instance, understanding the principles of aerodynamics could be a valuable resource for someone aspiring to fly.

Finally, he recommended that I document all these elements and iterate over them repeatedly. This process, he suggested, could lead to a deeper understanding of one's true desires.

Intrigued by this unconventional approach from a financial advisor, I inquired about its origin. He informed me that it was his original concept, and to his knowledge, no similar methodology existed elsewhere.

I spent several years looking for the same idea or a concept somewhere on the internet the one that I found recently is something that has similar name but differs in techique\footnote{https://support.infrontanalytics.com/en/support/solutions/articles/77000430396-what-is-gprv-} that I am about to work out. I realised later that what if these variables could be utilised for a profound path finding or could be used in application on data or some kind of algortihem that we can utilise for resolving say paths but it was harder to put out in something that could be implemted in generalised problem solving. I realised that this method could be useful in many cases for solving and why i keep mentioning that path finding meaning if someone is stuck at some problem or that has some form of a goal associated a problem wihout a goal is not a problem then.

I tried writing this down in different forms and some I am going to present you to make sure we are sticking to the practical side of it without getting too much in technicality. I am not sure if i need to qoute some literature or something before, I am not a conventional researcher. 

\section{VRPG as solver}

Consider a practical scenario wherein you aspire to purchase a car in 2023. This endeavor necessitates financial expenditure for the car's purchase, fuel, and maintenance. Furthermore, you may be environmentally conscious and concerned about the carbon dioxide emissions from the car. This example, while commonplace, is relevant to many individuals. Applying the VRPG (Values, Resources, Problems, Goals) model to this situation may present some complexities, but for the sake of clarity and understanding, let's proceed with defining some concepts and conventions.

Let's denote:

	\[ 
	G= Goals 
	\]
	\[ 
	P = Set \ of Problems 
	\]
	\[ 
	R = Set \ of Resources 
	\]
	\[ 
	V = Set \ of Values 
	\]

With these variables defined, we can assign values to them. For instance, 

	\[ 
	G= The \ aspiration \ to  \ purchase  \ a  \ car \ in \ 2023 
	\]

The next step involves identifying potential problems. These could include factors such as limited parking space or narrow roads, particularly if you reside in a place like Portugal. Another potential problem could be the lack of driving skills, which is a plausible issue in Portugal, where only 459 out of 1000 people own cars.

Thus, we have:

	\[ 
	P= {P(1), P(2)..P(n)} 
	\]
	
	\[ 
	P= Lack \ of \ driving \ skills, limited \ parking \ space , financial \ constraints
	\]

Problems can be both qualitative and quantitative. Qualitative problems lack measurable attributes, while quantitative problems can be measured. For instance, limited parking space and financial constraints could fall under both categories.

Next, we identify the resources available. These could include financial means such as savings, loans, or credit. It's important to note that these variables can be dynamic and may interchange with problems or values.

	\[ 
	R= {R(1), R(2)..R(n)} 
	\]
	
	\[ 
	R= Stable \ monthly\ income, \ parental \ support, \ ability\  to \ work
	\]

Finally, we consider the values that influence our decision. For instance, despite having the means to purchase a car, you might be committed to reducing carbon emissions and not contributing to the proliferation of vehicles on the planet.

	\[ 
	V= {V(1), V(2)..V(n)} 
	\]
	
	\[ 
	V= Desire \ to \ reduce \ CO2 \ emissions, \ reluctance \ to \ contribute \ to \ vehicle \ proliferation. 
	\]

This comprehensive framework allows us to analyze and understand the various factors that influence our decisions and actions.

\subsection{Identifying Objects}

Right now, it's feasible to say that there are possible ways to reduce text or sentences into some form of classes so in our case the classes are as stated above. These classes operate freely at this point we develop a function that tags these classes once we have them. Lets deep dive into a simple method such as calculating a sum of two values in our case.

To apply the concept of ``calculate x'' using the variables resources (R), values (V), and problems (P) as  described, we need to develop a mathematical model that accurately represents the interactions and impacts of these variables on the goal (X).

Variables:

\begin{itemize}
    \item Resources \( R \): Represent positive contributions towards achieving the goal.
    \begin{itemize}
        \item Example: \{1, 2, 3\}
    \end{itemize}
    \item Values \( V \): Represent factors that can either contribute positively or negatively, depending on the context.
    \begin{itemize}
        \item Example: \{1, 2, 3\}
    \end{itemize}
    \item Problems \( P \): Represent hindrances or negative impacts on achieving the goal.
    \begin{itemize}
        \item Example: \{1, 2, 4\}
    \end{itemize}
\end{itemize}

\textbf{Mathematical Representation:}

\begin{itemize}
    \item Positive Contribution \( R \): Assign a positive value to each element in \( R \). For instance, if \( R = \{1, 2, 3\} \), each number could represent a certain positive value towards \( X \).
    \item Negative Contribution \( P \): Assign a negative value to each element in \( P \). In the example \( P = \{1, 2, 4\} \), each number would detract from the goal \( X \).
    \item Swapping Contribution \( V \): Values in \( V \) can swap between adding to \( R \) or detracting as part of \( P \), depending on certain conditions or thresholds. This requires a rule or function to determine when a value in \( V \) contributes positively or negatively.
\end{itemize}

\textbf{Calculating \( X \):}

To calculate \( X \), which is the goal, we can construct a function that integrates these variables. One possible approach is:

\begin{enumerate}
    \item Initial Value of \( X \): Start with \( X = 0 \).
    \item Add Resources: Increment \( X \) by the sum of the values in \( R \).
    \item Subtract Problems: Decrement \( X \) by the sum of the values in \( P \).
    \item Apply Values: For each value in \( V \), determine if it adds to \( R \) or detracts as part of \( P \), and then adjust \( X \) accordingly.
\end{enumerate}

\textbf{Example Calculation:}

Let's say \( R = \{1, 2, 3\} \), \( V = \{1, 2, 3\} \), and \( P = \{1, 2, 4\} \).

\begin{itemize}
    \item For simplicity, assume each number represents its own value in the calculation.
    \item Assume a rule for \( V \): if a value is even, it adds to \( R \); if odd, it adds to \( P \).
\end{itemize}

Following this:

\begin{itemize}
    \item \( R \) becomes \( \{1, 2, 3, 2\} \) (adding 2 from \( V \)).
    \item \( P \) becomes \( \{1, 2, 4, 1, 3\} \) (adding 1 and 3 from \( V \)).
\end{itemize}

Then calculate \( X \):

\begin{align}
  X &= \text{sum}(R) - \text{sum}(P) \\
  &= (1+2+3+2) - (1+2+4+1+3) \\
  &= 8 - 11 \\
  &= -3
\end{align}

In this example, \( X = -3 \), indicating that the problems and negative aspects (P and negative contributions from V) outweigh the resources and positive contributions.

The above steps are simple but there are several considerations that needs to be thought out properly The model needs clear rules for how values in V are assigned to either R or P.
The values in R, V, and P need to be quantifiable in a way that makes sense for the goal X.
The model should be flexible to accommodate different scenarios and rules for V.
Sensitivity analysis might be required to understand how changes in R, V, and P impact X. The specifics of the model would depend on the exact nature of X and the context in which these variables operate.

Now to develop a model that translates user input regarding personal values, resources, and problems into a quantifiable goal (X), we can refine the approach by considering Values (V) as either quantifiers or qualifiers that can become resources or problems based on their positive or negative nature. Basing on the initial model.

\subsection{User Input and calculation model}

\begin{enumerate}
  \item \textbf{User Input:}
  \begin{itemize}
    \item Resources \( R \): Quantifiable positive factors (e.g., time, money, skills).
    \item Values \( V \): A mix of quantifiable and qualitative factors, each with a potential positive or negative impact.
    \item Problems \( P \): Quantifiable negative factors (e.g., financial constraints, time limitations).
  \end{itemize}

  \item \textbf{Quantifying Values \( V \):}
  \begin{itemize}
    \item Assign a positive or negative value to each element in \( V \) based on whether it’s a positive or negative influence.
    \item Positive Influence: Adds to resources (e.g., dedication, network).
    \item Negative Influence: Adds to problems (e.g., uncertainty, lack of experience).
  \end{itemize}

  \item \textbf{Example User Input Conversion:}
  \begin{align*}
    R &: \{5 \text{ (time)}, 3 \text{ (money)}\} \\
    V &: \{\text{dedication} (+2), \text{uncertainty} (-1), \text{creativity} (+1)\} \\
    P &: \{2 \text{ (financial constraint)}, 3 \text{ (time limitation)}\}
  \end{align*}

  \item \textbf{Mapping \( V \) to \( R \) and \( P \):}
  \begin{itemize}
    \item Add positive elements of \( V \) to \( R \) and negative elements to \( P \).
    \item New \( R \): \{5 \text{ (time)}, 3 \text{ (money)}, 2 \text{ (dedication)}, 1 \text{ (creativity)}\} = \{5, 3, 2, 1\}
    \item New \( P \): \{2 \text{ (financial constraint)}, 3 \text{ (time limitation)}, 1 \text{ (uncertainty)}\} = \{2, 3, 1\}
  \end{itemize}

  \item \textbf{Calculating Goal \( X \):}
  \begin{itemize}
    \item Define \( X \) as the net outcome of \( R \) and \( P \).
    \item \( X = \sum(R) - \sum(P) \).
  \end{itemize}

  \item \textbf{Performing Calculation with Example:}
  \begin{align*}
    X &= \sum(\{5, 3, 2, 1\}) - \sum(\{2, 3, 1\}) \\
      &= (5 + 3 + 2 + 1) - (2 + 3 + 1) \\
      &= 11 - 6 \\
      &= 5
  \end{align*}
  In this example, \( X = 5 \), indicating a positive net outcome based on the user’s resources, values, and problems.
\end{enumerate}

\section*{Considerations and Extensions}
\begin{itemize}
  \item User Input Flexibility: The model should accommodate a wide range of inputs, allowing for diverse user circumstances.
  \item Value Weighting: Depending on the context, different values \( V \) might have different weights or impacts.
  \item Dynamic Adjustment: Users’ values and circumstances can change over time, so the model should be adaptable.
  \item Goal Interpretation: The interpretation of \( X \) should be contextual. For example, a high positive \( X \) might indicate a strong likelihood of achieving a goal, while a negative \( X \) might suggest the need for reassessment or additional resources.
\end{itemize}

This model provides a framework for users to input their personal values, resources, and problems, translating them into a quantifiable measure towards achieving a specific goal. It can serve as a tool for personal or professional development planning, helping users to understand and navigate their unique circumstances. 

As we started out with the example in the begining of our curious case of bird that wishes to make \$ 100,000 a month. Are we ready to plug the data and find out.


\subsection{Nuanced Understanding}

In the preceding discourse, it has been elucidated that the computation of values utilizing rudimentary mathematical principles presents a relatively straightforward endeavor. However, the transposition of these calculations to real-world applications is notably more challenging, often necessitating a significant degree of optimization. A pivotal question arises regarding the certainty with which user inputs can be categorically assigned to distinct classifications such as 'Problems' within the linguistic domain. While the symbolic representation of problems is readily identifiable and broadly understood, there exists a tendency towards oversimplification in interpretation. 

Consider the illustrative instance of the aspiration to purchase an aircraft. When this objective is input into the previously discussed model, it may yield a positive outcome, predicated on numerous presumptions. The core inquiry here is whether it is feasible to effectively flatten these variables and reduce them to quantifiable measures. This reductionist approach seeks to transform qualitative aspirations into quantitative metrics, yet the extent to which this is achievable without oversimplification remains a subject of consideration. The challenge lies in preserving the complexity and nuance of real-world scenarios while endeavoring to represent them within a structured mathematical framework.

\section*{Matrix and Vector Approach to Modeling}

Adopting a matrix and vector approach for the model adds a layer of complexity and allows for a more nuanced understanding of how resources, values, and problems interact to influence the goal \( X \).
\subsection*{Step 1: Define Matrices and Vectors}
\begin{itemize}
  \item Resources (\( R \)), Values (\( V \)), and Problems (\( P \)) as matrices or vectors, where each element represents a different aspect or factor.
  \item Example:
  \begin{align*}
    R &= \begin{bmatrix} r_1 & r_2 & \cdots & r_n \end{bmatrix} \\
    V &= \begin{bmatrix} v_1 & v_2 & \cdots & v_n \end{bmatrix} \\
    P &= \begin{bmatrix} p_1 & p_2 & \cdots & p_n \end{bmatrix}
  \end{align*}
  \item Assigning Values to \( V \): Each value in \( V \) can be either positive (adding to \( R \)) or negative (adding to \( P \)). This can be represented as a transformation matrix or function that maps \( V \) to either \( R \) or \( P \).
\end{itemize}

\subsection*{Step 2: Matrix and Vector Transformations}
\begin{itemize}
  \item Transformation of \( V \): Apply a transformation to \( V \) that splits it into positive and negative components, adding them to \( R \) and \( P \) respectively.
  \item Example:
  \begin{align*}
    \text{Positive } V &= \begin{bmatrix} v_{1+} & v_{2+} & \cdots & v_{n+} \end{bmatrix} \text{ added to } R. \\
    \text{Negative } V &= \begin{bmatrix} v_{1-} & v_{2-} & \cdots & v_{n-} \end{bmatrix} \text{ added to } P.
  \end{align*}
  \item Revised \( R \) and \( P \): After transformation, you get new \( R \) and \( P \) vectors.
  \begin{itemize}
    \item New \( R \): Original \( R \) vector plus positive components of \( V \).
    \item New \( P \): Original \( P \) vector plus negative components of \( V \).
  \end{itemize}
\end{itemize}

\subsection*{Step 3: Calculating Goal (\( X \))}
\begin{itemize}
  \item Vector Calculation: Calculate \( X \) as a function of the distance or difference between the revised \( R \) and \( P \) vectors. This could be the Euclidean distance.
  \item Euclidean Distance Example:
  \[ X = \sqrt{\sum_{i}(r_i - p_i)^2} \]
  where \( r_i \) and \( p_i \) are elements of the revised \( R \) and \( P \) vectors.
\end{itemize}

\subsection*{Step 4: Interpreting \( X \)}
\begin{itemize}
  \item Positive \( X \): Indicates a greater influence of resources and positive values over problems.
  \item Negative \( X \): Suggests that problems and negative values outweigh resources.
  \item Magnitude of \( X \): Represents the overall balance or imbalance between resources, values, and problems.
\end{itemize}

\subsection*{Considerations}
\begin{itemize}
  \item Dimensionality: Ensure \( R \), \( V \), and \( P \) have the same dimensions for mathematical coherence.
  \item Scaling and Weighting: Different elements might need scaling or weighting based on their relative importance or impact.
  \item Contextual Interpretation: The meaning of \( X \) should be interpreted in the context of the specific goal or situation.
\end{itemize}

Using matrices and vector spaces in this way allows for a more sophisticated analysis of the relationships and influences among resources, values, and problems, providing a clearer picture of how they collectively impact the achievement of a goal.

\section*{Matrix and Vector Model Application to Aircraft Purchase}

To apply the matrix and vector model to the goal of buying an aircraft, we first need to define our variables - Resources (R), Values (V), and Problems (P) - in the context of this goal. Then we can perform a simple calculation using these elements.

\subsection*{Defining Variables}
\textbf{Resources (R):} These might include available budget, access to financing, existing aviation knowledge, etc.\\
\textit{Example:} \( R = [\text{budget}, \text{financing}, \text{knowledge}] = [50, 30, 20] \)

\textbf{Values (V):} These could be personal or business motivations such as passion for flying, convenience, or business needs.\\
\textit{Example:} \( V = [\text{passion}, \text{convenience}, \text{business utility}] = [20, 15, 25] \)

\textbf{Problems (P):} These are the challenges or negative aspects such as ongoing costs, regulatory challenges, and maintenance.\\
\textit{Example:} \( P = [\text{ongoing costs}, \text{regulations}, \text{maintenance}] = [40, 30, 20] \)

\subsection*{Transforming Values (V)}
Decide how each value in \( V \) contributes positively or negatively. For simplicity, let's assume all values in \( V \) are positive and add to \( R \).\\
\textit{Revised R = Original R + V =} \( [50, 30, 20] + [20, 15, 25] = [70, 45, 45] \)

\subsection*{Calculating Goal (X)}
Calculate \( X \) as the Euclidean distance between the revised \( R \) and \( P \).
\[ X = \sqrt{(r_1 - p_1)^2 + (r_2 - p_2)^2 + (r_3 - p_3)^2} \]

Using our example values:
\[ X = \sqrt{(70 - 40)^2 + (45 - 30)^2 + (45 - 20)^2} \]
\[ X = \sqrt{(30)^2 + (15)^2 + (25)^2} \]
\[ X = \sqrt{900 + 225 + 625} \]
\[ X = \sqrt{1750} \]

\subsection*{Performing the Calculation}
Let's calculate this in Python for an accurate result.\\	
The calculated value of \( X \) in the context of buying an aircraft is approximately 41.83.

\begin{lstlisting}[language=Python]

	import math

# Defining the variables
        R = [70, 45, 45]  # Revised Resources (original R + V)
        P = [40, 30, 20]  # Problems

# Calculating X as the Euclidean distance between R and P
       X = math.sqrt(sum((r - p) ** 2 for r, p in zip(R, P)))
       X	
	\end{lstlisting}

\subsection*{Interpretation:}
\textbf{Positive Value:} Since \( X \) is positive, it suggests that the combined resources and positive values outweigh the problems.\\
\textbf{Magnitude:} The magnitude (41.83) indicates the extent to which resources and values surpass the challenges.

This result can be interpreted as a generally favorable situation for achieving the goal of buying an aircraft, with the resources and positive values (like passion, convenience, and business utility) having a stronger impact than the identified problems (ongoing costs, regulations, maintenance). However, the actual decision should also consider qualitative aspects and personal circumstances beyond what the model can quantify.


\section{From Qualifiers to Quantifiers}

In the preceding model, the generation of numerical values from ostensibly qualitative data presents an intriguing conundrum. One might ponder the rationale behind assigning a specific numeric value, such as 20, to an abstract concept like 'knowledge.' This process, ostensibly, is predicated on a functional mechanism that ascertains these values. While methodologies such as Natural Language Processing (NLP) offer potential pathways for this determination, their complexity warrants a more simplified initial approach. Mathematical systems proffer diverse methodologies to assimilate subjective data or abstract elements, such as personal satisfaction or brand reputation.

The standard procedure involves the development of a weight-based scaling system. This technique entails mapping subjective assessments onto a numerical scale. Prevalent approaches include the Likert scale, encompassing a range of 1-5 or 1-10, where each numeral represents a specific degree of concurrence or intensity—1 signifying a minimal level and 10 indicating an extremely high level. An alternative approach is the Ordinal Scale, which involves ranking each element in order of its significance or impact.

Notwithstanding these methods, a significant question arises: where and how to effectively map this data, particularly in scenarios where such information is not directly provided by the user. In such cases, it becomes imperative to extrapolate and predict these values from textual data, subsequently constructing an index based on this analysis.

The situation inexorably leads us to the domain of Natural Language Processing. This field encompasses the analysis of sentiment, emotion recognition, offensiveness, and stance detection within text. An extensive review of relevant literature reveals various attempts and methodologies potentially akin to our objective. The journey towards employing large language models appears to be an unavoidable trajectory, especially in light of the reliance on textual data for this quantification process.


\section{Related Work}

In our case its important to understand what 

\section{Conclusion}
As demonstrated above to reduce down our understanding of a bird suddenly in flight decides to make money may not be the right one but it gets the message right. A larger question that needs to be answered at this point is how do we translate these into machine and make a general purpose solving using the 4 key variables that could be helpful in detection given if we have enough data and values then it all fits in the schema. In the next paper we are going to work on surveying the Deep Learning methods , transformers to reduce down the complexity and reduce text to numbers to our benefit that we could use to plot the goals and other values around it. We have tried our luck using some basic math and tried to uderstand that if its possible to calculate simple additions but to be able to 
\end{document}